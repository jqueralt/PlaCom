\documentclass[fontsize=10pt,%
paper=a4,%
%BCOR=12mm,% marge per relligar
%DIV=calc,%
DIV=14,%
%twoside,%per defecte 1 cara
pagesize=auto,%
%parskip=full,% no sagna paràgrafs però dóna més espai entre paràgrafs
parskip=half,
%parskip=false,
captions=tableheading,%
numbers=noenddot,%
toc=graduated%
%toc=flat %toc=flat,%índex amb ítems sense sagnar
]{scrartcl}
%]{scrreprt}
%--------------- configuració bàsica -----------------

\usepackage[utf8]{inputenc}
\usepackage[T1]{fontenc}
\usepackage{amsmath}
\usepackage{fix-cm} %per fer lletres gegants

\usepackage[catalan]{babel}                    % patrons de separació de paraules
%%%%%%%%%%%%%%%%%%%%% Guionatge
%
\hyphenation{} %llista de paraules separades per espais en blancc
%\babelhyphenation[catalan]{co-l·la-bo-ra-ció}


\usepackage[scaled=.88]{beramono}             % Bera-Monospace
\usepackage[scaled=.86]{berasans}             % Bera Sans-Serif
\renewcommand\familydefault{\sfdefault}

\usepackage{pifont}
%\usepackage{fouriernc}
%\usepackage{tgheros}
%\usepackage[scaled=.9]{helvet}
%\usepackage[osf]{mathpazo}                    % Palatino com a font per defecte
%\linespread{1.05}\selectfont                  % Palatino necessita més espai, aquí el 5%
%\usepackage{ifsym}                            % paquet per fer lletra de calculadora
%%% colors
\usepackage[pdftex,%					paquet per fer aparèixer els colors
dvipdfm,%
dvipsnames,%
svgnames,%
usenames,%
table]{xcolor}
%\usepackage[dvipsnames]{color}  % paquet de 68 colors
\definecolor{gris}{gray}{0.45}                   %gris
\definecolor{vermell}{rgb}{0.75,0.00,0.00}       %vermell
\definecolor{blau}{rgb}{0.16,0.24,0.64}       %blau
\definecolor{verd}{rgb}{0.18,0.58,0.08}       %verd
\definecolor{linkcolor}{rgb}{0,0,0.42}        %color de l'enllaç
\usepackage{blindtext}

%------------ paquets addicionals --------------
%%%%%%%%%%%%%%%%%%%%%%%% millores d'edició de versions
\usepackage[catalan,shadow,textwidth=1cm]{todonotes}

\usepackage{pdfpages}
%%%%%%%%%%%%% llistes niuades ------------------

%%% https://tex.stackexchange.com/questions/450272/resume-enumeration-in-sublist-with-enumitem-package/450274#450274
\usepackage[shortlabels]{enumitem}
\newlist{innerenumerate}{enumerate}{1}
\setlist*[innerenumerate]{label=\textbf{Acció\ \arabic*}, resume=inner,align=left}

%%%%%%%%%%%%%%%%%%%%%%%%%%%%%%%



\usepackage{xspace}                            % Espai automàtic després de macros
\usepackage{booktabs,array}                    % Taules agradables
\usepackage{ctable}
\usepackage{longtable}
\usepackage{multicol}
\usepackage{colortbl}                          % paquet per acolorir taules
\usepackage{tabularx}
\usepackage{multirow}
\usepackage{mdwlist}							%paquet per comprimir llistes
%\usepackage{adjustbox}
\usepackage{supertabular}
%\usepackage{xtab}

\usepackage[catalan]{varioref}
%\usepackage{textcomp}                         % senyal addicional
%\usepackage{varwidth}                         % per millorar el medi ambient
\usepackage{calc}                              % per càlculs
\usepackage{amsmath}
\usepackage{graphicx}
\usepackage{eso-pic}                           % paquet per clavar imatges a pàgines


%%%%%%%%%%%%%%%%%%%%%%% variables
\newcommand{\versio}{1}
\newcommand{\codi}{PlaCom}
\newcommand{\autor}{Joan Queralt Gil}
\newcommand{\curs}{2018-19}
\newcommand{\titol}{Pla de comunicació del centre}


%--------------------------- Formatatge KOMA---------------------------

%\addtokomafont{pagenumber}{\color{gray}}	             % dóna color als text del peu (números de pàgina+text)
%\addtokomafont{pagehead}{\normalfont\sffamily}          % dóna color al text de l'encapçalament
\addtokomafont{section}{\Large\sffamily\color{blau}}	 % dóna color a les Seccions
\addtokomafont{subsection}{\large\sffamily\color{blue}}	 % dóna color a les subseccions
%\addtokomafont{pagehead}{\normalfont\sffamily\small\color{gray}}   % fa que l'encapçalament/peu de la pàgina sigui sans serif i petit i no itàlica
\addtokomafont{subsubsection}{\color{blue}}


\addtokomafont{caption}{\footnotesize\itshape}
\addtokomafont{captionlabel}{\upshape\bfseries}
\addtokomafont{descriptionlabel}{\rmfamily}
\setcapindent{1em}
\deffootnote[1.2em]{1.7em}{1em}{\makebox[1.2em][l]{\thefootnotemark}}
%\deffootnote{2.25em}{1.75em}{\thefootnotemark\enspace}

%%% titulars de secció
% \renewcommand*{\othersectionlevelsformat}[1]{%
%   \makebox[22pt][l]{\csname the#1\endcsname\autodot}\enskip}
\reversemarginpar
%%aquesta redefinició fa que els números de secció surtin per l'esquerra del text
\renewcommand*{\othersectionlevelsformat}[1]{%
\makebox[0pt][r]{\csname the#1\endcsname\autodot\enskip}}

%%%% encapçalaments i peus %%%%%%%%%%%%%%%%%%%%%%%%%%%%%%%%%%%%%%%%%%%%%%%%%%%%%%%%%%%%%%%%%

\usepackage[automark]{scrlayer-scrpage} %automark posa automàticament la marca que correspon
\automark[subsection]{section}
\usepackage{lastpage}                        % fa el recompte de pàgines del document
\clearscrheadfoot
\pagestyle{scrheadings}
% fem servir el \pagestyle{scrheadings}
\setlength{\headheight}{25pt}% assigna 25 punts a l'alçada de la capçalera per fer espai
\setheadsepline{.4pt}% crea una línia sota la capçalera
\addtokomafont{headsepline}{\color{lightgray}}	% dóna color gris a la línia sota capçalera

%%% per treure el número de capítol/secció només a l'encapçalament
%\renewcommand*{\chaptermarkformat}{}
%\renewcommand*{\sectionmarkformat}{}

%%%%%%%%%%%%%%%%%%%%%%%%%%%%%%%%%%%%%%%%%
% per canviar de color els números de secció amb versions noves de Koma
% http://tex.stackexchange.com/questions/252627/section-numbers-colored
%\renewcommand*{\sectionformat}{%
%  \textcolor{gris}{\thesection}\autodot\enskip%
%}
%\renewcommand*{\subsectionformat}{%
%  \textcolor{LightSteelBlue}{\thesubsection}\autodot\enskip%
%}
% amb la versió vella de Koma:
%\renewcommand*{\othersectionlevelsformat}[3]{%
%\textcolor{gris}
%{#3}\autodot\enskip}



%%%%%%%%%%%%%%%% Capçaleres i peus%%%%%%%%%%%%%%%%%%%%%%%%%%%%%%%%%%%%
% fem servir el \pagestyle{scrheadings}
\ohead{}                                                     %capçalera exterior
\chead{\headmark}                                                              %capçalera centre
%peu centre:
\cfoot[
\begin{center}\footnotesize\sffamily
\renewcommand{\arraystretch}{1.75}%
\begin{tabular}{|p{1.2cm}|p{1,8cm}|p{7.3cm}|p{2.6cm}|} \hline
\multirow{2}{*}{\includegraphics[scale=1.15]{logo_IOC_blanc.jpg}} & Versió: \versio & {\normalsize{Pla de comunicació}} & Pàgina {\thepage} de  {\pageref{LastPage}} \\ \cline{2-2}\cline{4-4} & Codi: \codi     & {\normalsize{del centre}}         & {curs \curs}\\ \hline
\end{tabular}
{\color{gray}Assegureu-vos que aquest document és vigent.} Consulteu el web.
\end{center}
]
{
\begin{center}\footnotesize\sffamily
\renewcommand{\arraystretch}{1.75}% \begin{tabular}{|p{1.2cm}|p{2cm}|p{7.5cm}|p{2.2cm}|} \hline
\begin{tabular}{|p{1.2cm}|p{1,8cm}|p{7.3cm}|p{2.6cm}|} \hline
\multirow{2}{*}{\includegraphics[scale=1.15]{logo_IOC_blanc.jpg}} & Versió: \versio & {\normalsize{Pla de comunicació}} & Pàgina {\thepage} de  {\pageref{LastPage}} \\ \cline{2-2}\cline{4-4} & Codi: \codi     & {\normalsize{del centre}}                    & {curs \curs}\\ \hline
\end{tabular}
{\color{gray}Assegureu-vos que aquest document és vigent. Consulteu el web.}
\end{center}}
%\ofoot{\normalfont\normalcolor{\thepage\ |\  \pageref{LastPage}}}      %peu exterior



%%%%%%%%%%%%%%%%%%%%%%%%% configuració d'hyperref per pdf
\usepackage{hyperref}
\def\figureautorefname{Figura}
\def\tableautorefname{Taula}

\pdfstringdef{\Titol}{\titol}
\pdfstringdef{\Tema}{Pla de comunicació del centre}%
\pdfstringdef{\Claus}{Pla, comuniació, programació, IOC, responsables, tasques}%

\hypersetup{%
pdftitle=\Titol,%
pdfsubject=\Tema,%
pdfkeywords=\Claus,%
pdfauthor={\copyright 2019 Joan Queralt},%
%bookmarksopen=true, %finestreta de marcadors oberta quan s'obre el document
%backref=true,
bookmarksopenlevel=2,
linktocpage=true,
urlcolor=linkcolor,
citecolor=linkcolor,
linkcolor=linkcolor,
colorlinks=true,
}


\usepackage[
open,
openlevel=3,
atend
]{bookmark}[2016/05/17]
\bookmarksetup{color=blue}

\usepackage[stretch=10]{microtype}

\usepackage{lipsum}
%%%%%%%%%%%%%%%%%%%%%%%%%%%%%%%%%%%%%%%%%%%%%%%%%%%%%%%%%
\begin{document}

%%% portada %%%%%%%%%%%%%%%%%%%

%----------------------------- Pàgina del títol ----------------------------
\begin{titlepage}\sffamily
%\pdfbookmark[level]{bookmark text}{anchor}
\pdfbookmark[1]{Títol}{tit} %situa un punt de referència pels marcadors del document PDF

\begin{center}
 
\begin{minipage}{.8\linewidth}
\raggedright
%\raggedleft%
\noindent
% \makebox[0cm][l]{\scalebox{3}[-3]{\textcolor{blue!30}{Boxes}}}Boxes
{\color{blau}%\rule{\linewidth}{1pt}%
%\vspace{2\baselineskip}
\vspace{8\baselineskip}
{\Huge\bfseries
%Pla de comunicació\\%[1ex]

%del centre\\[3ex]

{\fontsize{60}{70}\selectfont Pla de comunicació}\\[6ex]

\curs \\
\vspace{2\baselineskip}
\noindent
%\rule{\linewidth}{1pt}
}}%
\end{minipage}
\end{center}



\vspace{4\baselineskip}
%\raggedleft
\raggedright
\vspace{\baselineskip}
\begin{center}
\let\footnoterule\relax
\begin{minipage}{.8\linewidth}\large
%Text a minipage
Codi: \codi\\
Versió: \versio\\
Curs:  \curs \\[3ex]
{\color{gray}\small{Assegureu-vos que aquest document és vigent. Consulteu el web.}} \\
\end{minipage}
\end{center}
\vfill
\bigskip
%\makebox[0cm][l]{\scalebox{1}[-1]{\textcolor{blue!30}{Boxes}}}Boxes
\AddToShipoutPicture*
{%
\put(100,100)%
{\includegraphics[scale=2]{logo_IOC_blanc.jpg}}%[scale=0.2]
}

\end{titlepage}

%------------------------ Final de la pàgina del títol ----------------------------
\clearpage % imports the commands from filename.tex into the target file; it's equivalent to typing all the commands

%%% pàgina de l'Índex %%%%%%%%

\thispagestyle{empty}
\pdfbookmark[1]{Índex}{toc}
\setcounter{tocdepth}{4}
\tableofcontents

%\clearpage
%\listoftodos %activar quan l'edició del doc
\clearpage
%%%%%%%%%%%%%%%%%%%%%%%%%%%%%%%

%%% primera pàgina del text
\pagestyle{scrheadings}


\section{Introducció}\label{sec:intro}

Aquest \textit{Pla de comunicació} de l'IOC recull i explica com s'ha d'organitzar la comunicació de la informació al centre. L'objectiu és acabar establint una comunicació fluïda, eficient i eficaç entre el centre i la comunitat educativa, tant la interna com l'externa.

El Pla de comunicació depèn del procés estratègic \textbf{PE04 Gestionar la informació, la comunicació i el coneixement}, la responsabilitat del qual és de la direcció.

En el darrer anàlisi DAFO realitzat sobre el centre, un dels aspectes més destacables que en sortí va ser les oportunitats que s'obririen amb una millora de la fluïdesa de la comunicació dins de la pròpia organització, però sobretot per donar a conèixer a l'exterior tot allò que estem fent i podem fer a l'IOC.

Hem analitzat les necessitats de comunicació de la organització amb cadascun dels grups d'interès per elaborar un pla de comunicació que s'hi adapti. A l'hora
d'establir aquest pla, s'han adequat els canals de comunicació al col·lectiu al qual s'adreça el missatge i activar els més efectius perquè sigui accessible a tothom.

Atès que l'IOC transmet missatges de moltes formes i per molts canals, ha de ser curós amb la imatge corporativa que transmet (ordre, neteja, identificació,
coherència, compromisos, resultats, ètica i estètica) perquè sigui la que realment persegueix el centre.

Disposem d'un logo corporatiu i d'una \textit{Guia d'estil} per les seves comunicacions internes i externes.

Finalment, la norma ISO 9001:2015 en l'apartat 7.4 de comunicació ens indica la necessitat que l'IOC determini en aquest Pla les comunicacions internes i externes relacionades amb el sistema de gestió de qualitat. Per això inclou:

\begin{itemize}
\item  què es comunicarà: \textit{Missatge}
\item  qui comunica: \textit{Responsable}
\item  a qui es comunica: \textit{Destinataris}
\item  com es comunica: \textit{Canal}
\item  quan es comunicarà: \textit{Moment}
\item  on queda registrat: \textit{Registre}
\end{itemize}

\section{Eixos de comunicació}\label{sec:eixos}

\subsection{Imatge de centre}

Un dels punts de millora que hem d'abordar com a centre de manera prioritària és
donar a conèixer a la comunitat l'existència i les activitats de l'IOC.

Tenim un seguit d'objectius i accions al pla de qualitat relacionats amb aquest punt i agrupats en l'Eix de Millora de la projecció de la imatge de l'IOC. Per això cal parar-hi força atenció i dedicar-li una cura especial.


\small{
\setlength{\tabcolsep}{10pt}
\renewcommand{\arraystretch}{1.5}
\begin{longtable}{p{3cm}p{2cm}p{2cm}p{1.5cm}p{2cm}p{1.5cm}}
\hline
\textbf{Missatge}                          & \textbf{Responsable} & \textbf{Destinataris} & \textbf{Canal} & \textbf{Moment}     & \textbf{Registre} \\
\hline \endfirsthead
\hline
\textbf{Missatge}                          & \textbf{Responsable} & \textbf{Destinataris} & \textbf{Canal} & \textbf{Moment}     & \textbf{Registre} \\
\hline \endhead
Informació general sobre el centre         & Equip directiu       & Comunitat             & Portal web     & Sempre              & Portal            \\
Notícies pròpies                           & Direcció             & Comunitat             & Portal web     & Quan es produeixen  & Portal            \\
Notícies pròpies                           & Direcció             & Comunitat             & Twitter        & Quan es produeixen  & @ioc              \\
Documentació estratègica i d'interès       & Direcció             & Comunitat             & Portal web     & Sempre              & Portal            \\
Projectes en els quals participa o impulsa & Equip directiu       & Comunitat             & Portal web     & Quan hi hagi canvis & Portal            \\
\end{longtable}
}%tanca small

\subsection{Difusió externa}


\small{
\setlength{\tabcolsep}{10pt}
\renewcommand{\arraystretch}{1.5}
\begin{longtable}{p{3cm}p{2cm}p{2cm}p{1.5cm}p{2cm}p{1.5cm}}
\hline
\textbf{Missatge}                   & \textbf{Responsable}   & \textbf{Destinataris}     & \textbf{Canal} & \textbf{Moment}          & \textbf{Registre}   \\
\hline \endfirsthead
\hline
\textbf{Missatge}                   & \textbf{Responsable}   & \textbf{Destinataris}     & \textbf{Canal} & \textbf{Moment}          & \textbf{Registre}   \\
\hline \endhead
Informació sobre l'oferta formativa & Equip directiu         & Comunitat                 & Portal web     & Sempre                   & Portal              \\
Recursos i material per l'estudi    & Equip directiu         & Estudiants                & Portal web     & Sempre                   & Portal              \\
Calendaris d'inscripció             & Equip directiu         & Comunitat                 & Portal web     & Quan es publiqui al DOGC & Portal              \\
Requisits de matrícula              & Equip directiu         & Comunitat                 & Portal web     & Quan canviïn             & Portal              \\
Incidències del servei              & Equip directiu         & Comunitat                 & Portal web     & Quan s'esdevinguin       & Portal              \\
Licitacions de serveis              & Direcció               & Comunitat                 & Portal web     & Sempre                   & Portal              \\
Reunions amb centres de suport      & Direccions acadèmiques & Responsables dels centres & Fòrum          & Quan correspongui        & Convocatòria i acta \\
\end{longtable}
}%tanca small



\subsection{Comunicació interna}


\small{
\setlength{\tabcolsep}{10pt}
\renewcommand{\arraystretch}{1.5}
\begin{longtable}{p{3cm}p{2cm}p{2cm}p{1.5cm}p{2cm}p{1.5cm}}
\hline
\textbf{Missatge}                        & \textbf{Responsable}     & \textbf{Destinataris}                    & \textbf{Canal} & \textbf{Moment}      & \textbf{Registre}                                          \\
\hline \endfirsthead
\hline
\textbf{Missatge}                        & \textbf{Responsable}     & \textbf{Destinataris}                    & \textbf{Canal} & \textbf{Moment}      & \textbf{Registre}                                          \\
\hline \endhead
Informació externa per correu postal     & Equip directiu           & Persones interessades                    & Calaix         & Sempre que arribi    & Cada servei                                                \\
Informació externa per correu electrònic & Equip directiu           & Persones interessades                    & e-Bústia       & Sempre que arribi    & Correu electrònic                                          \\
Convocatòria de Claustre                 & Direcció                 & Professorat                              & Fòrum          & Sempre que n'hi hagi & Convocatòria i acta                                        \\
Reunió d'Equip directiu                  & Direcció                 & Equip directiu                           & Fòrum          & Quan correspongui    & Convocatòria i acta                                        \\
Reunió Comissió de qualitat              & Direcció                 & Equip directiu i coordinació de qualitat & Fòrum          & Quan correspongui    & Convocatòria i acta                                        \\
Reunió de revisió del sistema            & Direcció                 & Comissió de qualitat                     & Fòrum          & Quan correspongui    & Convocatòria i acta                                        \\
Reunions d'Equips docents                & Direccions acadèmiques   & Professorat corresponent                 & Fòrum          & Quan correspongui    & Convocatòria i acta                                        \\
Reunions de coordinacions                & Direccions acadèmiques   & Professorat corresponent                 & Fòrum          & Quan correspongui    & Convocatòria i acta                                        \\
Reunions de tutors                       & Coordinacions de tutoria & Professorat corresponent                 & Fòrum          & Quan correspongui    & Convocatòria i acta                                        \\
Juntes d'avaluació                       & Direccions acadèmiques   & Professorat corresponent                 & Fòrum          & A final de període   & Convocatòria, informes dels centres de suport (GES) i acta \\
\end{longtable}
}%tanca small


\subsection{Participació en xarxes}


\small{
\setlength{\tabcolsep}{10pt}
\renewcommand{\arraystretch}{1.5}
\begin{longtable}{p{3cm}p{2cm}p{2cm}p{1.5cm}p{2cm}p{1.5cm}}
\hline
\textbf{Missatge} & \textbf{Responsable} & \textbf{Destinataris} & \textbf{Canal} & \textbf{Moment}    & \textbf{Registre} \\
\hline \endfirsthead
\hline
\textbf{Missatge} & \textbf{Responsable} & \textbf{Destinataris} & \textbf{Canal} & \textbf{Moment}    & \textbf{Registre} \\
\hline \endhead
Notícies pròpies  & Coordinacions        & Comunitat             & Twitter        & Quan es produeixen & @gesalioc

@miniops\_ioc

@ErasmusIoc

@fpinf\_ioc

@IOC\_PACFGS\_MA

@dep\_prp\_IOC                                                                                                             \\
\end{longtable}
}%tanca small


\subsection{Comunicació en moments de crisi}

En situacions complexes i urgents haurem d'establir quines actuacions es duran a terme que contribueixin a millorar la situació, sempre amb calma, transparència i organització. L'objectiu és mirar d'estalviar queixes, rumors i
males interpretacions de situacions que poden generar conflicte i anticipar-ne les solucions.

La informació que s'ha de donar en moments així ha de ser seriosa, transparent, verídica i clara. El silenci mai no és rendible, cal informar i cal fer-ho des del primer moment però, sobretot, cal que la informació sigui rigorosa, intel·ligible per part dels receptors, transparent i sobretot verídica. Potser no podrem explicar tota la veritat, però allò que diguem ha de ser cert.

Qui ha de donar resposta en moments així és l'equip directiu del centre, i
serà la Direcció qui ha de comandar la comunicació de les crisis.

El/la portaveu és una figura clau en la gestió d'una crisi. Ha de transmetre
confiança, proximitat, seguretat i credibilitat. Saber qui ha de parlar, què ha
de dir i quan ha de dir-ho és essencial per transmetre adequadament els
missatges durant una crisi.



\small{
\setlength{\tabcolsep}{10pt}
\renewcommand{\arraystretch}{1.5}
\begin{longtable}{p{3cm}p{2cm}p{2cm}p{1.5cm}p{2cm}p{1.5cm}}
\hline
\textbf{Missatge}  & \textbf{Responsable} & \textbf{Destinataris} & \textbf{Canal} & \textbf{Moment}      & \textbf{Registre} \\
\hline \endfirsthead
\hline
\textbf{Missatge}  & \textbf{Responsable} & \textbf{Destinataris} & \textbf{Canal} & \textbf{Moment}      & \textbf{Registre} \\
\hline \endhead
Caiguda del servei & Direcció             & Comunitat             & Portal web     & Sempre que succeeixi & Portal            \\
Caiguda del servei & Direcció             & Equip directiu        & Correu xtec

whatsapp

telèfon particular
                   & Sempre que succeeixi & El que correspongui                                                               \\
Endarreriment del pagament del professoat col·laborador & Direcció & Prof. col·laborador & Fòrum de les Sales & Quan es produeixi & Entrada al fòrum\\
\end{longtable}
}%tanca small


\section{Recursos i infraestructures}\label{sec:recursos}

\subsection{Canals de comunicació}

Correu corporatiu xtec i eines Google.

Campus virtual basat en Moodle i Mahara.

Portal web de comunicació amb la comunitat.

Xarxes socials: Twitter, Linkedin,\dots

\subsection{Infraestructures}

El funcionament de l'IOC està basat en unes infraestructures que li donen suport en el propi funcionament així com en la comunicació interna i externa, com ara són:

\begin{itemize}
\item Xarxa informàtica cablejada i cobertura wifi a totes les plantes de la seu
\item Correu corporatiu de la xarxa Xtec (*@xtec.cat)
\item En bústies externes correu corporatiu del centre (*@ioc.cat)
\item Moodle com a entorn virtual d'aprenentatge
\item Mahara com a base pel portafolis digital
\item Joomla com a base pel portal web ioc.xtec.cat
\item Dokuwiki per generar i publicar materials formatius
\item Pàgines consultives (web del departament d‟Ensenyament):
      \begin{itemize}
      \item XTEC: xtec.cat
      \item GENCAT: gencat.cat/educacio
      \item EACAT: idp.eacat.net
      \end{itemize}
\end{itemize}

\section{Execució i avaluació}\label{sec:execiaval}

L'avaluació del Pla de Comunicació té com a objectiu mesurar la percepció de
millora en la comunicació per part dels usuaris (professors, professors col·laboradors, PAS o centres de suport.

Utilitzarem diferents indicadors:


\small{
\setlength{\tabcolsep}{10pt}
\renewcommand{\arraystretch}{1.5}
\begin{longtable}{p{10.5cm}p{1.7cm}p{1.7cm}}
\hline
\textbf{Indicador}                                                             & \textbf{Responsable} & \textbf{Moment} \\
\hline \endfirsthead
\hline
\textbf{Indicador}                                                             & \textbf{Responsable} & \textbf{Moment} \\
\hline \endhead
1: Grau de satisfacció del professorat respecte a la comunicació               & Direcció             & Final de curs   \\
2: Grau de satisfacció del professorat col·laborador respecte a la comunicació & Direcció             & Final de curs   \\
3: Grau de satisfacció del PAS respecte a la comunicació                       & Direcció             & Final de curs   \\
4: Grau de satisfacció dels centres de suport respecte a la comunicació        & Direcció             & Final de curs   \\
\end{longtable}
}%tanca small



%%%%%%%%%%%%%%%%%%%%%%%%%%%%%%%%%%% Control de canvis

\section{Control de canvis}

\begin{description}
\item[Versió 00] Creació del document presentada per a la seva revisió i aprovació al novembre 2016.
\item[Versió 01] S'ha canviat el nom Procediment per Pla.\\ S'han afegit els eixos de comunicació.\\ S'han afegit les seccions: Imatge de centre, Participació en xarxes i Comunicació en moments de risc.\\ S'introdueixen els indicadors en l'avaluació del Pla.
\end{description}





%%%%%%%%%%%%%%%%%%%%%%%%%%%%%%%%%%% Gestió del document
\section{Gestió del document}\label{sec:gestiodoc}


\begin{center}\scriptsize\sffamily
\renewcommand{\arraystretch}{1.75}%
\begin{tabular}{lllp{4cm}}\hline
              & Elaborat per & Revisat per             & Aprovat per  \\ \hline
Nom i cognoms & Joan Queralt & Mercè Vidal             & Joan Queralt \\
Càrrec        & Director     & Coordinació de Qualitat & Director     \\
Data          & 08/01/2019   & 08/01/2019              & 08/01/2019   \\\hline
Signatura     &              &                         &
\end{tabular}
\end{center}

\end{document}

%%%%%%%%%%%%%%%%%%%%%%%%%%%%%%%%%%%%%%%%%%%%%%%%%%%%%%%%%%%%%%%%%%%%%%%%%%%%%%%%%%%%%%%%%%%%

\todo[inline]{revisar redactat}


\begin{center}

\small
\setlength{\tabcolsep}{10pt}
\renewcommand{\arraystretch}{1.5}
\begin{longtable}{p{3cm}p{2cm}p{2cm}p{1.5cm}p{2cm}p{1.5cm}}
\hline
\textbf{Missatge}                  & \textbf{Responsable} & \textbf{Destinataris} & \textbf{Canal} & \textbf{Moment} & \textbf{Registre} \\
\hline \endfirsthead
\hline
\textbf{Missatge}                  & \textbf{Responsable} & \textbf{Destinataris} & \textbf{Canal} & \textbf{Moment} & \textbf{Registre} \\
\hline \endhead
Informació general sobre el centre & Equip directiu       & Comunitat             & Portal web     & Sempre          & Portal            \\
\end{longtable}
\end{center}
